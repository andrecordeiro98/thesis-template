
%%%%%%%%%%%%%%%%%%%%%%%%%%%%%%%%%%%%%%%%%%%%%%%%%%%%%%%%%%%%%%%%%%%%%%%%%%%%%%%
% PACKAGES
%%%%%%%%%%%%%%%%%%%%%%%%%%%%%%%%%%%%%%%%%%%%%%%%%%%%%%%%%%%%%%%%%%%%%%%%%%%%%%%

%Language
\usepackage[utf8]{inputenc} 			% Input encoding
\usepackage[english]{babel} 			% English
\usepackage{csquotes}                   % ADDED to make sure quoted texts show up correctly.

%Mathematics
\usepackage{amsmath}					% American Mathematical Society Package
\usepackage{amsfonts}					% Mathematical Fonts
\usepackage{amssymb}					% More mathematical symbols
\usepackage{mathtools}					% More control and better appearance of Mathematics

%Figures and captions
\usepackage{graphicx}					% More complex version of graphics for figures
%\usepackage{caption}					% More control over captions
\usepackage[labelfont=bf,font=small]{caption}		% More control over captions (and boldface "Figure 1.1." text)
\usepackage[labelfont=bf,font=small]{subcaption}	% Same, for subcaptions
\usepackage{sidecap}					% Side captions (just in case)

\pdfsuppresswarningpagegroup=1			% This just suppresses an annoying (and meaningless) warning

%ThE LooOks!
\usepackage{fancyhdr}					% Fancy Headers
\usepackage{geometry}					% Control geometry
\usepackage{setspace}
\usepackage[bottom]{footmisc}           %Send footnotes to the bottom of the page

%Tables and other environments
\usepackage{tabularx}					% Better tables
\usepackage{booktabs}					% Even better tables
\usepackage{enumitem}					% More control over enumerate environments
\usepackage{arydshln}					% Use dashed lines in tables

%Physics
\usepackage{siunitx}					% SI units			
\usepackage{physics}					% Useful notations (bras and kets for example)

% Useful Tools
\usepackage{lipsum}						% Create random text
\usepackage{comment}					% Create comments
%\usepackage{xcolor} 					% Colors
\usepackage[dvipsnames]{xcolor} 		% Better Colors
\usepackage{framed}						% Colored frames
\usepackage{setspace}					% Control spaces

%My packages
\usepackage{slashed}                    % Feynman notation for Dirac matrices, i.e. \slashed{}
\usepackage{nccmath}                    % More math stuff (\mfrac for medium sized fractions!)
\usepackage{bbold}                      % For \mathbb{1}
\usepackage{multirow}                   % multirows
\usepackage{booktabs}                   % nice tables, idk
\usepackage{environ}                    %\NewEnviron{}{}{} to define environments

%Load last
\usepackage{hyperref}					% Create hyperlinks


%%%%%%%%%%%%%%%%%%%%%%%%%%%%%%%%%%%%%%%%%%%%%%%%%%%%%%%%%%%%%%%%%%%%%%%%%%%%%%%
% DOCUMENT SETUP
%%%%%%%%%%%%%%%%%%%%%%%%%%%%%%%%%%%%%%%%%%%%%%%%%%%%%%%%%%%%%%%%%%%%%%%%%%%%%%%

%Paper Geometry
\geometry{
	paper=a4paper, 	% A4 Paper
	inner=2.5cm, 	% Inner margin
	outer=2.5cm, 	% Outer margin
	top=2.5cm, 		% Top margin
	bottom=2.5cm, 	% Bottom margin
}

%Line spacing
\renewcommand{\baselinestretch}{1.5}


% ---------------------------
% SELECT THE MATH FONT
% NOTE: Comment out all of them for the default computer modern 

%\usepackage{mathpazo}	% Loads Palatino for math use (serifed math font)
\usepackage{cmbright}	% Loads Computer Modern Bright (normal text font can be different)

% ---------------------------
% SELECT THE NORMAL TEXT FONT

%\newcommand{\fontnamestring}{phv}	% Helvetica
\newcommand{\fontnamestring}{cmbr}	% Computer Modern Bright

\renewcommand{\rmdefault}{\fontnamestring}
\renewcommand{\sfdefault}{\fontnamestring}

\def\FontLn{% 16 pt normal
	\usefont{T1}{\fontnamestring}{m}{n}\fontsize{16pt}{16pt}\selectfont}
\def\FontLb{% 16 pt bold
	\usefont{T1}{\fontnamestring}{b}{n}\fontsize{16pt}{16pt}\selectfont}
\def\FontMn{% 14 pt normal
	\usefont{T1}{\fontnamestring}{m}{n}\fontsize{14pt}{14pt}\selectfont}
\def\FontMb{% 14 pt bold
	\usefont{T1}{\fontnamestring}{b}{n}\fontsize{14pt}{14pt}\selectfont}
\def\FontSn{% 12 pt normal
	\usefont{T1}{\fontnamestring}{m}{n}\fontsize{12pt}{12pt}\selectfont}




%------------------------
%-------- MINE ----------
%------------------------
%No adjustment, all blank space is sent to the bottom of the page
\raggedbottom


%%%%%%%%%%%%%%%%%%%%%%%%%%%%%%%%%%%%%%%%%%%%%%%%%%%%%%%%%%%%%%%%%%%%%%
% NOTE: THESE COMMANDS (ALL THREE) MAKE SURE THE MATH MODE IS BIGGER
%		I DIDN'T USE THEM, BUT IT MAY VERY WELL LOOK NICER

%Change math mode size
%\DeclareMathSizes{<tfs>}{<ts>}{<ss>}{<sss>}
% <tfs> --- Surrounding text
% <ts>  --- Mathmode text
% <ss>  --- Subscript
% <sss> --- Subsubscript

%%%%%%%%%%%%%%%%%%%%%%%%%%%%%%%%%%%%%%%%%%%%%%%%%%%%%%%%%%%%%%%%%%%%%%

% UNCOMMENT THE FOLLOWING THREE LINES
\DeclareMathSizes{9}{10}{8}{6}
\DeclareMathSizes{10}{10}{8}{6}
\DeclareMathSizes{11}{10}{8}{6}

%--------------------------
% Enabling \mathscr{} for pretty calligraphic fonts. :)

\makeatletter
\DeclareFontEncoding{LS1}{}{}
\DeclareFontEncoding{LS2}{}{\noaccents@}
\DeclareFontSubstitution{LS1}{stix}{m}{n}
\DeclareFontSubstitution{LS2}{stix}{m}{n}
\makeatother

\DeclareMathAlphabet\mathscr{LS1}{stixscr}{m}{n}
\SetMathAlphabet\mathscr{bold}{LS1}{stixscr}{b}{n}
%--------------------------


%--------------------------
% Small bold for CMBright

\DeclareFontFamily{OT1}{cmbr}{\hyphenchar\font45 }
\DeclareFontShape{OT1}{cmbr}{m}{n}{%
	<-9>cmbr8
	<9-10>cmbr9
	<10-17>cmbr10
	<17->cmbr17
}{}
\DeclareFontShape{OT1}{cmbr}{m}{sl}{%
	<-9>cmbrsl8
	<9-10>cmbrsl9
	<10-17>cmbrsl10
	<17->cmbrsl17
}{}
\DeclareFontShape{OT1}{cmbr}{m}{it}{%
	<->ssub*cmbr/m/sl
}{}
\DeclareFontShape{OT1}{cmbr}{b}{n}{%
	<->ssub*cmbr/bx/n
}{}
\DeclareFontShape{OT1}{cmbr}{bx}{n}{%
	<->cmbrbx10
}{}

%--------------------------




%------------------------
%------------------------


%Make tables align at the separator '.'
\usepackage{dcolumn}
\newcolumntype{d}{D{.}{.}{-1}}

%URL links setup
\colorlet{url_blue}{blue!50!black}

\hypersetup{
    %pdftitle            = {Thesis Title},
	pdfpagemode			= {UseOutlines}	,
	bookmarksopen		= true 			,
	bookmarksopenlevel	= 0				,
	hypertexnames		= false			,
	colorlinks			= true			, % Set to false to disable coloring links
	citecolor			= blue			, % The color of citations
	linkcolor			= blue			, % The color of references to document(sections, figures, etc)
	urlcolor			= url_blue		, % The color of hyperlinks (URLs)
	pdfstartview		= {FitV}			,
	breaklinks			= true,
	unicode								,
}

%Setup sidecaption aligned to top of figure and caption width
\sidecaptionvpos{figure}{t}
\captionsetup{width=.85\textwidth}
%\captionsetup{width=.95\textwidth}


%Set shaded color
\definecolor{shadecolor}{rgb}{0.8,0.8,0.8}

%%%%%%%%%%%%%%%%%%%%%%%%%%%%%%%%%%%%%%%%%%%%%%%%%%%%%%%%%%%%%%%%%%%%%%%%%%%%%%%
% COVER PAGE TOOLS
%%%%%%%%%%%%%%%%%%%%%%%%%%%%%%%%%%%%%%%%%%%%%%%%%%%%%%%%%%%%%%%%%%%%%%%%%%%%%%%

%new Latex variable names
\newcommand{\coverThesis}{@undefined} 
\newcommand{\coverSupervisors}{@undefined}
\newcommand{\coverExaminationCommittee}{@undefined}
\newcommand{\coverChairperson}{@undefined} 
\newcommand{\coverSupervisor}{@undefined} 
\newcommand{\coverMemberCommittee}{@undefined} 

\addto\captionsenglish{\renewcommand{\coverThesis}{Thesis to obtain the Master of Science Degree in}}
\addto\captionsenglish{\renewcommand{\coverSupervisors}{Supervisors}}
\addto\captionsenglish{\renewcommand{\coverExaminationCommittee}{Examination Committee}}
\addto\captionsenglish{\renewcommand{\coverChairperson}{Chairperson}}
\addto\captionsenglish{\renewcommand{\coverSupervisor}{Supervisor}}
\addto\captionsenglish{\renewcommand{\coverMemberCommittee}{Members of the Committee}}

%%%%%%%%%%%%%%%%%%%%%%%%%%%%%%%%%%%%%%%%%%%%%%%%%%%%%%%%%%%%%%%%%%%%%%%%%%%%%%%
% ACKNOWLEDGEMENT SECTION
%%%%%%%%%%%%%%%%%%%%%%%%%%%%%%%%%%%%%%%%%%%%%%%%%%%%%%%%%%%%%%%%%%%%%%%%%%%%%%%

% new LaTeX variable name
\newcommand{\acknowledgments}{@undefined} 
\addto\captionsenglish{\renewcommand{\acknowledgments}{Acknowledgements}}

%%%%%%%%%%%%%%%%%%%%%%%%%%%%%%%%%%%%%%%%%%%%%%%%%%%%%%%%%%%%%%%%%%%%%%%%%%%%%%%
% NOMENCLATURE AND GLOSSARY
%%%%%%%%%%%%%%%%%%%%%%%%%%%%%%%%%%%%%%%%%%%%%%%%%%%%%%%%%%%%%%%%%%%%%%%%%%%%%%%

%-----------------------
% If you want the nomenclature table (list of variables). NOT REQUIRED BY IST

\usepackage{nomencl}
\makenomenclature

% Group variables according to their symbol type
\RequirePackage{ifthen}

\renewcommand{\nomgroup}[1]{%
	\ifthenelse{	\equal{#1}{R}	}{	\item[\textbf{Roman symbols}]	}{%
	\ifthenelse{	\equal{#1}{G}	}{	\item[\textbf{Greek symbols}]	}{%
   	\ifthenelse{	\equal{#1}{S}	}{	\item[\textbf{Subscripts}]		}{%
  	\ifthenelse{	\equal{#1}{T}	}{	\item[\textbf{Superscripts}]	}{%   
  	}}}}}
    

%-----------------------
% For the list of Abbreviations (required by IST)

%Include 'nogroupskip' in options below to eliminate spacing between groups of acronyms starting with the same letter
%\usepackage[acronym,nonumberlist,nogroupskip]{glossaries} %Works fine for overleaf
\usepackage[acronym,indexonlyfirst,nonumberlist,nogroupskip,nomain]{glossaries} %Needed for TeXStudio (nomain == do not use main glossary) 
\makeglossaries


%%%%%%%%%%%%%%%%%%%%%%%%%%%%%%%%%%%%%%%%%%%%%%%%%%%%%%%%%%%%%%%%%%%%%%%%%%%%%%%
% BIBLIO SETUP
%%%%%%%%%%%%%%%%%%%%%%%%%%%%%%%%%%%%%%%%%%%%%%%%%%%%%%%%%%%%%%%%%%%%%%%%%%%%%%%

%NOTE: MONTHS SHOULD BE INTEGERS, USE month = {3} INSTEAD OF month = {mar}
%NOTE: 'biber' instead of 'bibtex' for bibliography to show up
%NOTE: 'style=chem-angew' has no titles in the Bibliography.
%NOTE: 'giveninits=true' writes authors' initials

%NOTE: sorting=<any of the following>
%nty—sorts entries by name, title, year;
%nyt—sorts entries by name, year, title;
%nyvt—sorts entries by name, year, volume, title;
%anyt—sorts entries by alphabetic label, name, year, title;
%anyvt—sorts entries by alphabetic label, name, year, volume, title;
%ynt—sorts entries by year, name, title;
%ydnt—sorts entries by year (descending order), name, title;
%none—no sorting. Entries appear in the order they appear in the text.

%NOTE: How to change how many authors appear in the bibliography
% maxbibnames=<value> (3 by default)
% minbibnames=<value> (1 by default)

%\usepackage[backend=bibtex, style=chem-angew, doi=false, url=false, isbn=false]{biblatex}
%\usepackage[backend=biber, style=chem-angew, doi=false, url=false, isbn=false]{biblatex}
\usepackage[backend=biber, giveninits=true, doi=false, url=false, isbn=false, sorting=none, maxbibnames=5]{biblatex}

%Ignore 'note' field
\AtEveryBibitem{%
  \clearfield{note}%
}

%Bibliography file
%\bibliography{./bib/my_ref} 

%ALWAYS specify the full path. Also, \bibliography is apparently deprecated, favouring \addbibresource
%HOWEVER, both work (!) (in Overleaf, that is)
%\bibliography{thesis-template/bib/my_ref} 
\addbibresource{thesis-template/bib/my_ref.bib}
%%%%%%%
% NOTE
% To get this to work on TeXStudio, do: 
%  Options > Configure TeXStudio > Build > Default Bibliography Tool = Biber 
%%%%%%%

%---------------------------------------------
% Set hyperlinks for the Bibliography entries
\newbibmacro{string+doiurlisbn}[1]{%
	\iffieldundef{url}{%
		\iffieldundef{doi}{%
			\iffieldundef{isbn}{%
				\iffieldundef{issn}{%
					#1%
				}{%
					\href{http://books.google.com/books?vid=ISSN\thefield{issn}}{#1}%
				}%
			}{%
				\href{http://books.google.com/books?vid=ISBN\thefield{isbn}}{#1}%
			}%
		}{%
			\href{http://dx.doi.org/\thefield{doi}}{#1}%
		}%
	}{%
		\href{\thefield{url}}{#1}%
	}%
}

\DeclareFieldFormat{title}{\usebibmacro{string+doiurlisbn}{\mkbibemph{#1}}}
\DeclareFieldFormat[article,incollection]{title}{\usebibmacro{string+doiurlisbn}{\mkbibquote{#1}}}



%%%%%%%%%%%%%%%%%%%%%%%%%%%%%%%%%%%%%%%%%%
% OLD CODE. IGNORE
%%%%%%%%%%%%%%%%%%%%%%%%%%%%%%%%%%%%%%%%%%

\mycomment{
\newbibmacro{string+doiurl}[1]{%
	\iffieldundef{url}
	{\iffieldundef{doi}
		{#1}
		{\href{https://doi.org/\thefield{doi}}{#1}}}
	{\href{\thefield{url}}{#1}}}


\makeatletter

\def\blx@driver#1{%
	\ifcsdef{blx@bbx@#1}
	{\usebibmacro{string+doiurl}{\csuse{blx@bbx@#1}}}
	{\ifcsdef{blx@bbx@*}
		{\blx@warning{%
				No driver for entry type '#1'.\MessageBreak
				Using fallback driver}%
			\usebibmacro{string+doiurl}{\csuse{blx@bbx@*}}}
		{\blx@error
			{No driver found}
			{I can't find a driver for the entry type
				'\abx@field@entrytype'\MessageBreak
				and there is no fallback driver either}}}}
\makeatother
}

