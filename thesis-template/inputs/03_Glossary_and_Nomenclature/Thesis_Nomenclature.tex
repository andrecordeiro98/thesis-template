% In case you want to define your own nomenclature groups
\renewcommand{\nomgroup}[1]{%
	\ifthenelse{	\equal{#1}{R}	}{	\item[\textbf{Roman symbols}]	}{%
	\ifthenelse{	\equal{#1}{G}	}{	\item[\textbf{Greek symbols}]	}{%
	\ifthenelse{	\equal{#1}{S}	}{	\item[\textbf{Subscripts}]		}{%
	\ifthenelse{	\equal{#1}{T}	}{	\item[\textbf{Superscripts}]	}{%   
	\ifthenelse{	\equal{#1}{Z}	}{	\item[\textbf{Other}]			}{%
}}}}}}

% Separation between symbol and description
\setlength{\nomlabelwidth}{1.5cm}


% The definitions can be placed anywhere in the document body
% and their order is sorted by <symbol> automatically when
% calling makeindex in the makefile
%
% The \glossary command has the following syntax:
%
% \glossary{entry}
%
% The \nomenclature command has the following syntax:
%
% \nomenclature[<prefix>]{<symbol>}{<description>}
%
% where <prefix> is used for fine tuning the sort order,
% <symbol> is the symbol to be described, and <description> is
% the actual description.

% ----------------------------------------------------------------------
% Roman symbols [r]

\nomenclature[r]{$T\ind{a}{ij}$}{SU(N) generator.}

\nomenclature[r]{$f\ind{abc}{}$}{SU(N) structure functions.}

%Phantoms to force position (alphabetical)
\nomenclature[r]{$\phantom{}u(p)$}{Spinor for fermions (incoming).}
\nomenclature[r]{$\phantom{}\bar{u}(p)$}{Spinor for fermions (outgoing).}

\nomenclature[r]{$\phantom{}\bar{v}(p)$}{Spinor for antifermions (incoming).}
\nomenclature[r]{$\phantom{}v(p)$}{Spinor for antifermions (outgoing).}

\nomenclature[r]{$\mathscr{L}$}{Lagrangian, or lagrangian density.}
% ----------------------------------------------------------------------
% Greek symbols [g]

\nomenclature[g]{$\mu,\nu,\sigma$}{Spacetime indices.}

\nomenclature[g]{$\varepsilon\sub{\mu}(p)$}{Polarisation for gluons (incoming).}
\nomenclature[g]{$\varepsilon\sub{\mu}(p)\upp{*}$}{Polarisation for gluons (outgoing).}

\nomenclature[g]{$\gamma\upp{\mu}$}{Dirac matrix.}

% ----------------------------------------------------------------------
% Subscripts [s]

\nomenclature[s]{$i,j,k$}{Colour indices pertaining to the fundamental representation.}

% ----------------------------------------------------------------------
% Supercripts [t]

\nomenclature[t]{$a,b,c$}{Colour indices pertaining to the adjoint representation.}

% ----------------------------------------------------------------------
% Others

\nomenclature[z]{$\partial\sub{\mu}$}{Partial derivative.}

