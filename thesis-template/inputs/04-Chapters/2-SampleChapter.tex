\chapter{A Sample Chapter}
\label{chapter:SampleChapter}

This chapter contains some examples of equations, figures, footnotes\footnote{So you can see the type size: according to regulation, it should be 9pt (corresponding to $\backslash \texttt{small}$ if the main text is 10pt). In addition to this, I am also writing a lot of extra text so you can see the line spacing of footnotes equals one line.} and tables with a suggested style.

\section{Some Text}

\lipsum[2]

%\clearpage
\section{Some Equations, even in section titles, \texorpdfstring{$E = h \nu$}{E = h ν}}

The following sections contain various examples of typeset equations, such as $\left(\partial\sub{\mu}\partial\upp{\mu} - m\upp{2}\right) \phi = 0$, or $f\sub{p/h}(x, Q\upp{2})$. It also shows how to use math in section titles, without problems in the table of contents.

%$\mfrac{\partial f\sub{i} (x,\mu\upp{2})}{\partial \ln \mu\upp{2}} = \mfrac{\alpha(\mu\upp{2})}{2\pi} \int\ind{x}{1} \diff z \, f\sub{j}(x,\mu\upp{2}) P\sub{ij}(\mfrac{x}{z})$ 

\subsection{Single Line, Single Label Equation}

\begin{align}
	\diff \ln \rho\upp{2} \, \diff z
	\,=\,
	{\rm det}
	\begin{bmatrix}
		\,1 & f'(z) / f(z) \\
		\,0 & 1
	\end{bmatrix} 
	\diff \ln \mu\upp{2} \, \diff z
	\,=\,
	\diff \ln \mu\upp{2} \, \diff z
	\,.
\end{align}

\subsection{Multiple Line, Single Label Equation}

\begin{align} \begin{split}
	\mathscr{L}\sub{\textsf{classical}} \,&=\, 
	\overline{\psi} ( \im \slashed{D} - {\sf M} ) \psi - \frac{1}{4} F\ind{a}{\mu\nu} F\ind{a,\mu\nu}{} \,,
	\quad\textrm{ with }\quad\\
	%%%%%%%%%%%%%%%%%%%%%%%%
	D\sub{\mu} \,&=\, \partial\sub{\mu} \,-\, \im\,g\, \textrm{T}\upp{a} \, A\ind{a}{\mu} \,,\\
	F\ind{a}{\mu\nu} \,&=\, \partial\sub{\mu} A\ind{a}{\nu} \,-\, \partial\sub{\nu} A\ind{a}{\mu}
	\,+\, g\, f\upp{abc}\,A\ind{b}{\mu}\,A\ind{c}{\nu}\,.
\end{split} \end{align}

\subsection{Multiple Line, Multiple Label Equation}

\begin{subequations} \begin{align}
		\mfrac{1}{(-g)} \mathcal{M}&\sub{\mathcal{B} \rightarrow q\overline{q}g }
		=
		\varepsilon\sub{\mu}(k)\upp{*} \,
		\overline{u}(p\sub{1})
		\textrm{T}\ind{a}{ij}
		\bigg(
		\gamma\upp{\mu}
		\frac{ \slashed{k} + \slashed{p}\sub{1} + m }{ (p\sub{1} + k)\upp{2} - m\upp{2} }
		\mathcal{B}
		\,-\,
		\mathcal{B}
		\frac{ \slashed{k} + \slashed{p}\sub{2} - m }{ (p\sub{2} + k)\upp{2} - m\upp{2} }
		\gamma\upp{\mu}
		\bigg)
		v(p\sub{2})
		\\
		%----------------------------------------
		=&\,
		\overline{u}(p\sub{1})
		\, \mathcal{B} \,
		v(p\sub{2}) \,
		\varepsilon\sub{\mu}(k)\upp{*} \,
		\bigg(
		\frac{ p\ind{\mu}{1} }{ p\sub{1}\cdot k }
		\,-\,
		\frac{ p\ind{\mu}{2} }{ p\sub{2}\cdot k }
		\bigg) \textrm{T}\ind{a}{ij}
		%----
		+ \overline{u}(p\sub{1})
		\bigg(
		\frac{ \slashed{\varepsilon}\upp{*} \, \slashed{k} \, \mathcal{B} }{ 2 \, p\sub{1}\cdot k }
		\,-\,
		\frac{ \mathcal{B} \, \slashed{k} \, \slashed{\varepsilon}\upp{*} }{ 2 \, p\sub{2}\cdot k }
		\bigg)
		v(p\sub{2})
		\textrm{T}\ind{a}{ij}
		\label{eqline:amplitude_antenna}
		\,.
\end{align} \label{eq:amplitude_antenna} 
\end{subequations}
%
Here we see equation \eqref{eq:amplitude_antenna}, with special emphasis on line \eqref{eqline:amplitude_antenna}.



%------------------------------

%\clearpage
\section{Some Figures}

\subsection{Single Figure}

\begin{figure}[!ht]
	\centering
	\includegraphics[width=.40\linewidth]{example-image-a}
	\caption{A figure, with a caption. A very long caption. A really rather long caption, to show off the line width for captions.}
	\label{fig1:single_image}
\end{figure}


%\clearpage
\subsection{Side By Side}

\begin{figure}[!ht]
	\centering
	\begin{subfigure}[t]{.40\textwidth}
		\centering
		\includegraphics[width=\linewidth]{example-image-a}
		\subcaption{A sub-figure.}
		\label{subfig1:left_subimage}
	\end{subfigure}
	\hspace{.10\textwidth} % some spacing. Alternatively, \hfill works well
	\begin{subfigure}[t]{.40\textwidth}
		\centering
		\includegraphics[width=\linewidth]{example-image-b}
		\subcaption{Another sub-figure.}
		\label{subfig1:right_subimage}
	\end{subfigure}
	\caption{Two different sub-figures.}
	\label{fig1:two_images}
\end{figure}

In \ref{fig1:two_images}, we have two images, \ref{subfig1:left_subimage} and \ref{subfig1:right_subimage}, side by side.

%\lipsum[4-5]

\clearpage
\subsection{Three Figures}

\begin{figure}[!ht]
	\begin{subfigure}[t]{.40\textwidth}
		\centering
		\includegraphics[width=\linewidth]{example-image-a}
		\subcaption{First.}
		\label{subfig1:first_image}
	\end{subfigure}
	\hspace{.10\textwidth}
	\begin{subfigure}[t]{.40\textwidth}
		\centering
		\includegraphics[width=\linewidth]{example-image-b}
		\subcaption{Second.}
		\label{subfig1:second_image}
	\end{subfigure}
	\centering
	\begin{subfigure}[t]{.40\textwidth}
		\centering
		\includegraphics[width=\linewidth]{example-image-c}
		\subcaption{Third.}
		\label{subfig1:third_image}
	\end{subfigure}
	\caption{Three sub-figures, in the same image.}
	\label{fig1:three_images}
\end{figure}

%\clearpage
\section{Some Tables}

\subsection{A Simple Table}

\begin{table}[ht!]
	\centering
	\begin{tabular}{c c c c c c c}
		\toprule
		\textbf{Measurement} & 
		$1$ & 	$2$ & 	$3$ & 	$4$ & 	$5$ & $6$ 
		\\
		%------------------------------
		\midrule
		Experiment 1 
		& \num{4.20}	
		& \num{27.13}	
		& \num{70.06}	
		& \num{133.35}	
		& \num{180.42}	
		& \num{267.73}	\\ 
		Experiment 2 
		& \num{6.34}	
		& \num{37.72}	
		& \num{99.00}	
		& \num{132.12}	
		& \num{201.62}	
		& \num{221.37}	\\ 
		Experiment 3 
		& \num{10.62}	
		& \num{64.53}	
		& \num{171.98}	
		& \num{282.91}	
		& \num{335.09}	
		& \num{541.63}  \\
		\bottomrule
	\end{tabular}
	\caption{Table with a simple structure.}
	\label{tab4:tauform_vals}
\end{table}



\subsection{A More Complicated Table}

\begin{table}[ht]
	\centering
	\begin{tabular}{c c c c c}
		\toprule
		\multirow{2}{*}{\textbf{Analysis Methods}} &
		\multicolumn{4}{c}{\textbf{Data Sets}} \\
		%------------------------------
		& 
		\textbf{Data Set 1} & 
		\textbf{Data Set 2} & 
		\textbf{Data Set 3} & 
		\textbf{Data Set 4} \\
		%------------------------------
		\midrule
		Method 1	& \num{5.09}	& \num{4.15}	& \num{6.02}	& \num{8.49}		\\ 
		Method 2	& \num{2.19}	& \num{2.09}	& \num{4.44}	& \num{4.74}		\\ 
		\bottomrule
	\end{tabular}
	\caption{Table with a somewhat less simple structure.}
	\label{tab2:tau_form_resolution_mean}
\end{table}
